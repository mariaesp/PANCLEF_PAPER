\documentclass{llncs}
\usepackage[american]{babel}
\usepackage[T1]{fontenc}
\usepackage{times}
\usepackage{graphicx}

%%%%%%%%%%%%%%%%%%%%%%%%%%%%%%%%%%%%%%%%%%%%%%%%%%%%%%%%%%%%%%%%%%%%%%%%
\begin{document}

\title{Analyzing User Profiles for Detection of Fake News Spreaders on Twitter}
%%% Please do not remove the subtitle.
\subtitle{Notebook for PAN at CLEF \the\year}

\author{María S. Espinosa \and Roberto Centeno \and Álvaro Rodrigo}
\institute{Departamento de Lenguajes y Sistemas Informáticos, Universidad Nacional de Educación a Distancia (UNED), Spain.\\
mespinosa@lsi.uned.es, alvarory@lsi.uned.es, rcenteno@lsi.uned.es}

\maketitle

\begin{abstract}
The massive spread of digital information to which our society is subjected nowadays has led to a great amount of false or extremely biased information being shared and consumed by Internet users every day. Disinformation, including misleading and even false information, is a major issue for our current society. The impact of fake news on politics, economy and health is yet to be specified. Internet users must face a high amount of false information in digital media such as rumours, fake news, and extremely biased news. Given the crucial role that the spread of fake news plays in our current society, it is becoming essential to design tools to automatically verify the veracity of online information. 

In order to address this issue, this work offers a detailed description of the system developed for the detection of fake news spreaders on Twitter. The model was evaluated at the \emph{Profiling Fake News Spreaders on Twitter} task on Author Profiling at the PAN@CLEF 2020 competition. The model approaches the problem of identifying fake news spreaders 
\end{abstract}


\section{Introduction}

The notebooks shall contain a full write-up of your approach, including all details necessary to reproduce your results.


Body of text. This should contain information on:  
tasks performed  
main objectives of experiments  
approach(es) used and progress beyond state-of-the-art  
resources employed  
results obtained  
analysis of the results  
perspectives for future work  


\bibliographystyle{splncs03}
\begin{raggedright}
\bibliography{}
\end{raggedright}

\end{document}


%%%%%%%%%%%%%%%%%%%%%%%%%%%%%%%%%%%%%%%%%%%%%%%%%%%%%%%%%%%%%%%%%%%%%%%%

